%%%%%%%%%%%%%%%%%%%%%%%%%%%%%%%%%%%%%%%%%
% The Legrand Orange Book
% LaTeX Template
% Version 2.0 (9/2/15)
%
% This template has been downloaded from:
% http://www.LaTeXTemplates.com
%
% Mathias Legrand (legrand.mathias@gmail.com) with modifications by:
% Vel (vel@latextemplates.com)
%
% License:
% CC BY-NC-SA 3.0 (http://creativecommons.org/licenses/by-nc-sa/3.0/)
%
% Compiling this template:
% This template uses biber for its bibliography and makeindex for its index.
% When you first open the template, compile it from the command line with the 
% commands below to make sure your LaTeX distribution is configured correctly:
%
% 1) pdflatex main
% 2) makeindex main.idx -s StyleInd.ist
% 3) biber main
% 4) pdflatex main x 2
%
% After this, when you wish to update the bibliography/index use the appropriate
% command above and make sure to compile with pdflatex several times 
% afterwards to propagate your changes to the document.
%
% This template also uses a number of packages which may need to be
% updated to the newest versions for the template to compile. It is strongly
% recommended you update your LaTeX distribution if you have any
% compilation errors.
%
% Important note:
% Chapter heading images should have a 2:1 width:height ratio,
% e.g. 920px width and 460px height.
%
%%%%%%%%%%%%%%%%%%%%%%%%%%%%%%%%%%%%%%%%%

%----------------------------------------------------------------------------------------
%	PACKAGES AND OTHER DOCUMENT CONFIGURATIONS
%----------------------------------------------------------------------------------------

\documentclass[11pt,fleqn,openany]{book} % Default font size and left-justified equations

%----------------------------------------------------------------------------------------

\input{structure} % Insert the commands.tex file which contains the majority of the structure behind the template

\begin{document}
%----------------------------------------------------------------------------------------
%	TITLE PAGE
%----------------------------------------------------------------------------------------

\begingroup
\thispagestyle{empty}
\begin{tikzpicture}[remember picture,overlay]
\coordinate [below=12cm] (midpoint) at (current page.north);
\node at (current page.north west)
{\begin{tikzpicture}[remember picture,overlay]
\node[anchor=north west,inner sep=0pt] at (0,0) {\includegraphics[width=\paperwidth]{Pictures/background.pdf}}; % Background image
\draw[anchor=north] (midpoint) node [fill=ocre!30!white,fill opacity=0.6,text opacity=1,inner sep=1cm]{\Huge\centering\bfseries\sffamily\parbox[c][][t]{\paperwidth}{\centering Stemmarest\\[15pt] % Book title
{\Large Dokumentation des PSE2 Projekt}\\[20pt] % Subtitle
{\small Jakob Schaerer, Severin Zumbrunn, Ido Gershoni, Joel Niklaus, Ramona Imhof}}}; % Author name
\end{tikzpicture}};
\end{tikzpicture}
\vfill
\endgroup

%----------------------------------------------------------------------------------------
%	COPYRIGHT PAGE
%----------------------------------------------------------------------------------------

\newpage
~\vfill
\thispagestyle{empty}

\noindent Copyright \copyright\ 2015 Team PSE2\\ % Copyright notice

\noindent \textsc{Published by ...}\\ % Publisher

\noindent \textsc{....org}\\ % URL

\noindent Licensed under the Creative Commons Attribution-NonCommercial 3.0 Unported License (the ``License''). You may not use this file except in compliance with the License. You may obtain a copy of the License at \url{http://creativecommons.org/licenses/by-nc/3.0}. Unless required by applicable law or agreed to in writing, software distributed under the License is distributed on an \textsc{``as is'' basis, without warranties or conditions of any kind}, either express or implied. See the License for the specific language governing permissions and limitations under the License.\\ % License information

\noindent \textit{First printing, March 2013} % Printing/edition date

%----------------------------------------------------------------------------------------
%	TABLE OF CONTENTS
%----------------------------------------------------------------------------------------

\chapterimage{header.png} % Table of contents heading image

\pagestyle{empty} % No headers

\tableofcontents % Print the table of contents itself

\cleardoublepage % Forces the first chapter to start on an odd page so it's on the right

\pagestyle{fancy} % Print headers again

%----------------------------------------------------------------------------------------
%	PART
%----------------------------------------------------------------------------------------

\part{Project}

%----------------------------------------------------------------------------------------
%	CHAPTER 1
%----------------------------------------------------------------------------------------

\chapterimage{header.png} % Chapter heading image

\chapter{Introduction}

Lorem ipsum dolor sit amet, consetetur sadipscing elitr, sed diam nonumy eirmod tempor invidunt ut labore et dolore magna aliquyam erat, sed diam voluptua. At vero eos et accusam et justo duo dolores et ea rebum. Stet clita kasd gubergren, no sea takimata sanctus est Lorem ipsum dolor sit amet. Lorem ipsum dolor sit amet, consetetur sadipscing elitr, sed diam nonumy eirmod tempor invidunt ut labore et dolore magna aliquyam erat, sed diam voluptua. At vero eos et accusam et justo duo dolores et ea rebum. Stet clita kasd gubergren, no sea takimata sanctus est Lorem ipsum dolor sit amet.


%----------------------------------------------------------------------------------------
%	CHAPTER 2
%----------------------------------------------------------------------------------------

\chapterimage{header.png} % Chapter heading image

\chapter{Database (neo4j)}

\section{Structure}

\begin{center}
\includegraphics[scale=0.65]{Pictures/db_overview.png} 
\end{center}

\begin{center}
\includegraphics[scale=0.65]{Pictures/db_overview2.png} 
\end{center}
%----------------------------------------------------------------------------------------
%	CHAPTER 3
%----------------------------------------------------------------------------------------

\chapterimage{header.png} % Chapter heading image

\chapter{Jersey}

Lorem ipsum dolor sit amet, consetetur sadipscing elitr, sed diam nonumy eirmod tempor invidunt ut labore et dolore magna aliquyam erat, sed diam voluptua. At vero eos et accusam et justo duo dolores et ea rebum. Stet clita kasd gubergren, no sea takimata sanctus est Lorem ipsum dolor sit amet. Lorem ipsum dolor sit amet, consetetur sadipscing elitr, sed diam nonumy eirmod tempor invidunt ut labore et dolore magna aliquyam erat, sed diam voluptua. At vero eos et accusam et justo duo dolores et ea rebum. Stet clita kasd gubergren, no sea takimata sanctus est Lorem ipsum dolor sit amet.

%----------------------------------------------------------------------------------------
%	CHAPTER 3
%----------------------------------------------------------------------------------------

\chapterimage{header.png} % Chapter heading image

\chapter{Performance}

The performance is tested by benchmark tests. Please consult the chapter Benchmark Tests in the part Testing of this documentation for further technical information.

\begin{figure}[h!]
  \caption{An example to show what will come here.}
  \centering
    \includegraphics[scale=0.6]{firstBenchmark.png}
\end{figure}


%----------------------------------------------------------------------------------------
%	PART
%----------------------------------------------------------------------------------------

\part{Testing}

%----------------------------------------------------------------------------------------
%	CHAPTER 1
%----------------------------------------------------------------------------------------

\chapterimage{header.png} % Chapter heading image

\chapter{Concept}
This chapter describes the test-concept of the Digital Humanities PSE2 Project. The testing is used to assure the quality of the project and for test driven development. All tests are written in a manner they don't have any impact on the architecture of the project. To achieve this object mocking and injection by mockito is used. 

\section*{Integration Tests}
Every user story is tested by an integration Test. The integration tests assures the quality of the project. The technique of integration tests is described in the Integration test chapter. 

\section*{Unit Tests}
Unit tests are used for test driven development and are defined by the developer. They are only for the development process and are not referenced in quality audits.

\newpage

\section*{Jersey Overview} 

\begin{center}
\includegraphics[scale=.40]{Pictures/jerseyoverview.png} 
\end{center}

In the productive system for every REST call jersey is instantiating the requested resource and providing the service. Each resource object has its own database service, which is closed after each call. In production a embedded neo4j GraphDatabase is used. To achieve a minimal invasive test system the productive database needs to be replaced with a test database. To change the database without test related code in the project, object injection is used. 


\chapter{Injection}
Mockito is used to mock, spy and inject objects in the tests. To use Mockito with junit the following annotation has to be done directly above the class header.
\begin{lstlisting}[language=java]
@RunWith(MockitoJUnitRunner.class)
\end{lstlisting}
With the MockitoJUnitRunner the annotations @Mock, @Spy and @InjectMocks can be used. To inject an GraphDatabaseService, the GraphDatabaseFactory is mocked and the newEmbeddedDatabase method overwritten that it returns an impermanent database. 
\begin{lstlisting}[language=java]
@Mock
protected GraphDatabaseFactory mockDbFactory =
   new GraphDatabaseFactory();
\end{lstlisting}
The Mock annotation creates a Mock object. This is needed to overwrite the newEmbeddedDatabase method.
\begin{lstlisting}[language=java]
Mockito.when(mockDbFactory.newEmbeddedDatabase(Matchers.anyString()))
	.thenReturn(mockDbService);  
\end{lstlisting}
Where mockDbService is a spy object. This is needed to suppress each shutdown call of this object. This is necessary because the resource is not new created for every restcall and so the database should not be closed. See more in the Integrationtest chapter. 
\begin{lstlisting}[language=java]
@Spy
protected GraphDatabaseService mockDbService = 
  new TestGraphDatabaseFactory().newImpermanentDatabase();
\end{lstlisting}

\begin{lstlisting}[language=java]
Mockito.doNothing().when(mockDbService).shutdown();
\end{lstlisting}

And the injection of the Mockobjects into a resource

\begin{lstlisting}[language=java]
@InjectMocks
private User userResource;
\end{lstlisting}

%----------------------------------------------------------------------------------------
%	CHAPTER 2
%----------------------------------------------------------------------------------------

\chapterimage{header.png} % Chapter heading image

\chapter{Unit Test}

\begin{center}
\includegraphics[scale=.4]{Pictures/junitoverview.png}
\end{center} 
 
For Unit Tests the methods of the resource are called directly.
\begin{lstlisting}[language=java]
@Test
public void SimpleTest(){
  String actualResponse = userResource.getIt();
  assertEquals(actualResponse, "User!");
}
\end{lstlisting}


\subsection*{Example}
\url{https://github.com/tohotforice/PSE2_DH/blob/e364fcb0c164981281c5799a6bf9f9f9ea5eb503/stemmarest/src/test/java/net/stemmaweb/stemmaserver/UserUnitTest.java}
%----------------------------------------------------------------------------------------
%	CHAPTER 3
%----------------------------------------------------------------------------------------

\chapterimage{header.png} % Chapter heading image

\chapter{Integration Tests}

\begin{center}
\includegraphics[scale=.4]{Pictures/jerseytestoverview.png} 
\end{center}

To inject objects into a resource it is mandatory that the resource is created statically at the place the injection is done. This is not possible when the resources are instantiated when a REST call occurs. To solve this JerseyTestServerFactory creates a server where already instantiated resources can be registered. 
To start a JerseyTestServer a global JerseyTest has to be created.
\begin{lstlisting}[language=java]
private JerseyTest jerseyTest;
\end{lstlisting}
The JerseyTestServerFactory creates a JerseyTest with already instantiated resources. This is necessary to inject the mock objects. Multiple resources can be added by chaining .addResource(..).addResource()
\begin{lstlisting}[language=java]
jerseyTest = JerseyTestServerFactory.newJerseyTestServer()
	.addResource(userResource).create();
jerseyTest.setUp();
\end{lstlisting}

The test is done by calling a webresource of jerseyTest
\begin{lstlisting}[language=java]
@Test
public void SimpleTest(){
  String actualResponse = jerseyTest.resource()
      .path("/user").get(String.class);
  assertEquals(actualResponse, "User!");
}
\end{lstlisting}

\subsection*{Example}
\url{https://github.com/tohotforice/PSE2_DH/blob/e364fcb0c164981281c5799a6bf9f9f9ea5eb503/stemmarest/src/test/java/net/stemmaweb/stemmaserver/UserTest.java}

%----------------------------------------------------------------------------------------
%	CHAPTER 4
%----------------------------------------------------------------------------------------

\chapterimage{header.png} % Chapter heading image

\chapter{Benchmark Testing}

A main goal of the PSE2 stemmarest project is a good performance compared to the previous RESTful service. To measure the performance benchmark testing is needed. A benchmark test basically calls the RESTful service multiple times and measure the response time. To achieve this \textit{com.carrotsearch.junitbenchmarks} a handy JUnit benchmark test suite is used. This JUnitbenchmarks measure the time which is used to execute a test and can generate visual representations of the measurement.\\
For the benchmark testing it is of interest to have a variety of different databases. Those databases should differ in their size from small to very huge. This allows to measure the algorithms in extreme situations.
To generate valid graphs only limited by diskspace the class \textit{RandomGraphGenerator} can be used. By calling the static method role a graph is generated according to the parameters.\\
\begin{remark}
Please note that the response time highly depends on the hardware the tests are running on and the actual state of Javas virtual machine. 
\end{remark}
To reduce the influence of the virtual machine before the measurements 5 warm-up calls are done to bring the virtual machine to live. The hardware which was used for testing is represented in the report.

\section*{Setup}
All the classes related to the Benchmark Tests can be found in the package \\ \textit{net.stemmaweb.stemmaserver.benachmarktests}. The class BenchmarkTests contains all Tests. The classes Benchmark<n>Nodes contain the database generation. Here is configured how many nodes the database has. This are also the classes which are run with JUnit test. BenchmarkTests cant be run as a JUnit test as it is a abstract class. s

\begin{center}
\includegraphics[scale=0.4]{BenchmarkTestsClassOverview.jpg} 
\end{center}

Tests are simply implemented in the BenchmarkTests class with the @Test annotation. It is best practice only to implement the restcall in this method and only test if Response.Status is OK. This assures that as less as possible overhead time is measured. And the integration- and JUnit tests should be done on a other place.

\begin{remark}
JUnitBenchmarks measures the time to execute (@Before, @Test, @After). Heavy operations which should not be measured can be done in @BeforeClass and @AfterClass.
\end{remark}

To create a new database test-environment copy the class Benchmark600Nodes and rename it to the count of Nodes that should be inserted. In the class itself only two small adjustments need to be done. First change the name of the report file \textit{@BenchmarkMethodChart(filePrefix = ''benchmark/benchmark-600Nodes'')}. Second adjust the properties of the database which should be generated \textit{rgg.role(db, 2, 1, 3, 100);}. role(databaseService, cardinalityOfUsers, cardinalityOfTraditionsPerUser, cardinalityOfWitnessesPerTradition, degreeOfTheTraditionGraphs)

\section*{Run Benchmarktests}
The Benchmarktests can be run as every JUnit test. But to generate the report an argument needs to be passed by. 
Create a JUnit Test as follows:

\begin{center}
\includegraphics[scale=0.4]{benchmarkTest.png} 
\end{center}

\begin{center}
\includegraphics[scale=0.4]{benchmarkTestArgument.png} 
\end{center}

On the tab Arguments \textit{-Djub.consumers=CONSOLE,H2 -Djub.db.file=.benchmarks} has to be inserted into the VM Arguments input. After the test can be run as usual. After the test execution the reports are stored under \textit{benchmark/}.
 
\begin{remark}
The execution of the tests will take some time because of the generation of huge graphs. Its recommended not to use the computer during this tests.  
\end{remark}


%----------------------------------------------------------------------------------------
%	PART
%----------------------------------------------------------------------------------------

\part{RESTful API}

%----------------------------------------------------------------------------------------
%	CHAPTER 1
%----------------------------------------------------------------------------------------

\chapterimage{header.png} % Chapter heading image

\chapter{Documentation}

\section{/user}
\begin{get}
/user
\end{get}

\subsection*{Summary}
Returns a welcome message
\begin{parameter}
\end{parameter}
\begin{return}[SUCCESS]
\textbf{text/plain}\\
''User!''
\end{return}

\section{/user/create}
\begin{post}
/user/create
\end{post}

\subsection*{Summary}
Creates a user.
\begin{parameter}
\textbf{application/json}\\
\{ ''userId'':<userId>, ''isPublic'':<isAdmin> \}
\end{parameter}
\begin{return}[CREATED]
\textbf{application/json}\\
\{ ''userId'':<userId>, ''isPublic'':<isAdmin> \}
\end{return}
\begin{return}[CONFLICT]
\textbf{application/json}\\
Error: A user with this id already exists
\end{return}

\section{/user/\{id\}}
\begin{get}
/user/\{id\}
\end{get}

\subsection*{Summary}
Returns the user as JSON Object
\begin{parameter}\textbf{URL}\\
 Id: the user id
\end{parameter}
\begin{return}[OK]
\textbf{application/json}\\
\{ 'userId': <userId>, 'isAdmin': <isAdmin> \}\\
\textit{The information about the user}
\end{return}
\begin{return}[NOT\_FOUND]
\textbf{application/json}\\
\textit{The information about the user}
\end{return}

\section{/user/traditions/\{userId\}}
\begin{get}
/user/traditions/\{userId\}
\end{get}

\subsection*{Summary}
List all Traditions of a user
\begin{parameter}\textbf{URL}\\
userId: the id of the user \\
\end{parameter}
\begin{return}[OK]
\textbf{application/json}\\
\{''traditions'':[ \{''name'':<traditionName> \} ] \}
\end{return}
\begin{return}[NOT\_FOUND]
\textbf{application/json}\\
Error: A user with this id does not exist!
\end{return}

\section{/textinfo/\{textId\}}
\begin{post}
/textinfo/\{textId\}
\end{post}

\subsection*{Summary}
Update the textInfo of a tradition.
\begin{parameter}\textbf{URL}\\
textId: the id of the tradition \\
\end{parameter}
\begin{parameter}
\textbf{application/json}\\
\{  'name': <new\_name>, 'language': <new\_language>, 'isPublic': <is\_public>, 'ownerId': <new\_ownerId>  \}
\end{parameter}
\begin{return}[SUCCESS]
\textbf{application/json}\\
\{  'name': <new\_name>, 'language': <new\_language>, 'isPublic': <is\_public>, 'ownerId': <new\_ownerId>  \}\\
\textit{The new information of the tradition.}
\end{return}
\begin{return}[CONFLICT]
\textbf{application/json}\\
''Error: A user with this id does not exist''\\
\textit{If the user does not exist. }
\end{return}
\begin{return}[NOT\_FOUND]
\textbf{application/json}\\
\textit{If the tradition was not found. }
\end{return}

\section{/tradition/witness/\{tradId\}}
\begin{get}
/tradition/witness/\{tradId\}
\end{get}

\subsection*{Summary}
List all Witness of a tradition
\begin{parameter}\textbf{URL}\\
tradId: the id of the tradition \\
\end{parameter}
\begin{return}[OK]
\textbf{application/json}\\

\end{return}
\begin{return}[NOT\_FOUND]
\textbf{application/json}\\
Error: A tradition with this id does not exist!
\end{return}

\section{/tradition/new}
\begin{post}
/tradition/new
\end{post}

\subsection*{Summary}
Create a new tradition.
\begin{parameter}
\textbf{text/plain}\\
name: The name of the tradition
\end{parameter}
\begin{parameter}
\textbf{multipart/form-data}\\
language: The language of the tradition\\
public: 0 if the tradition is not public 1 if the tradition is public\\
name: The name of the tradition\\
file: multipart file input stream
\end{parameter}
\begin{return}[CONFLICT]
\textbf{application/json}\\
''Error: No user with this id exists''
\end{return}
\begin{return}[INTERNAL\_SERVER\_ERROR]
\textbf{application/json}\\
''Error: Tradition could not be imported!''
\textit{If the server was not able to parse the input file}
\end{return}
\begin{return}[OK]
\textbf{application/json}\\
''Tradition imported successfully''
\end{return}

\section{/tradition/get/\{tradId\}}
\begin{get}
/tradition/get/\{tradId\}
\end{get}

\subsection*{Summary}
Get a graphml of a tradition
\begin{parameter}\textbf{URL}\\
tradId: the id of the tradition \\
\end{parameter}
\begin{return}[OK]
\textbf{application/xml}\\

\end{return}
\begin{return}[NOT\_FOUND]
\textbf{application/json}\\
Error: A tradition with this id does not exist!
\end{return}

%----------------------------------------------------------------------------------------
%	BIBLIOGRAPHY
%----------------------------------------------------------------------------------------

\chapter*{Bibliography}
\addcontentsline{toc}{chapter}{\textcolor{ocre}{Bibliography}}
\section*{Books}
\addcontentsline{toc}{section}{Books}
\printbibliography[heading=bibempty,type=book]
\section*{Articles}
\addcontentsline{toc}{section}{Articles}
\printbibliography[heading=bibempty,type=article]

%----------------------------------------------------------------------------------------
%	INDEX
%----------------------------------------------------------------------------------------

\cleardoublepage
\phantomsection
\setlength{\columnsep}{0.75cm}
\addcontentsline{toc}{chapter}{\textcolor{ocre}{Index}}
\printindex

%----------------------------------------------------------------------------------------

\end{document}